% Options for packages loaded elsewhere
\PassOptionsToPackage{unicode}{hyperref}
\PassOptionsToPackage{hyphens}{url}
%
\documentclass[
]{article}
\usepackage{lmodern}
\usepackage{amssymb,amsmath}
\usepackage{ifxetex,ifluatex}
\ifnum 0\ifxetex 1\fi\ifluatex 1\fi=0 % if pdftex
  \usepackage[T1]{fontenc}
  \usepackage[utf8]{inputenc}
  \usepackage{textcomp} % provide euro and other symbols
\else % if luatex or xetex
  \usepackage{unicode-math}
  \defaultfontfeatures{Scale=MatchLowercase}
  \defaultfontfeatures[\rmfamily]{Ligatures=TeX,Scale=1}
\fi
% Use upquote if available, for straight quotes in verbatim environments
\IfFileExists{upquote.sty}{\usepackage{upquote}}{}
\IfFileExists{microtype.sty}{% use microtype if available
  \usepackage[]{microtype}
  \UseMicrotypeSet[protrusion]{basicmath} % disable protrusion for tt fonts
}{}
\makeatletter
\@ifundefined{KOMAClassName}{% if non-KOMA class
  \IfFileExists{parskip.sty}{%
    \usepackage{parskip}
  }{% else
    \setlength{\parindent}{0pt}
    \setlength{\parskip}{6pt plus 2pt minus 1pt}}
}{% if KOMA class
  \KOMAoptions{parskip=half}}
\makeatother
\usepackage{xcolor}
\IfFileExists{xurl.sty}{\usepackage{xurl}}{} % add URL line breaks if available
\IfFileExists{bookmark.sty}{\usepackage{bookmark}}{\usepackage{hyperref}}
\hypersetup{
  pdftitle={Project - 1},
  pdfauthor={Randall T},
  hidelinks,
  pdfcreator={LaTeX via pandoc}}
\urlstyle{same} % disable monospaced font for URLs
\usepackage[margin=1in]{geometry}
\usepackage{color}
\usepackage{fancyvrb}
\newcommand{\VerbBar}{|}
\newcommand{\VERB}{\Verb[commandchars=\\\{\}]}
\DefineVerbatimEnvironment{Highlighting}{Verbatim}{commandchars=\\\{\}}
% Add ',fontsize=\small' for more characters per line
\usepackage{framed}
\definecolor{shadecolor}{RGB}{248,248,248}
\newenvironment{Shaded}{\begin{snugshade}}{\end{snugshade}}
\newcommand{\AlertTok}[1]{\textcolor[rgb]{0.94,0.16,0.16}{#1}}
\newcommand{\AnnotationTok}[1]{\textcolor[rgb]{0.56,0.35,0.01}{\textbf{\textit{#1}}}}
\newcommand{\AttributeTok}[1]{\textcolor[rgb]{0.77,0.63,0.00}{#1}}
\newcommand{\BaseNTok}[1]{\textcolor[rgb]{0.00,0.00,0.81}{#1}}
\newcommand{\BuiltInTok}[1]{#1}
\newcommand{\CharTok}[1]{\textcolor[rgb]{0.31,0.60,0.02}{#1}}
\newcommand{\CommentTok}[1]{\textcolor[rgb]{0.56,0.35,0.01}{\textit{#1}}}
\newcommand{\CommentVarTok}[1]{\textcolor[rgb]{0.56,0.35,0.01}{\textbf{\textit{#1}}}}
\newcommand{\ConstantTok}[1]{\textcolor[rgb]{0.00,0.00,0.00}{#1}}
\newcommand{\ControlFlowTok}[1]{\textcolor[rgb]{0.13,0.29,0.53}{\textbf{#1}}}
\newcommand{\DataTypeTok}[1]{\textcolor[rgb]{0.13,0.29,0.53}{#1}}
\newcommand{\DecValTok}[1]{\textcolor[rgb]{0.00,0.00,0.81}{#1}}
\newcommand{\DocumentationTok}[1]{\textcolor[rgb]{0.56,0.35,0.01}{\textbf{\textit{#1}}}}
\newcommand{\ErrorTok}[1]{\textcolor[rgb]{0.64,0.00,0.00}{\textbf{#1}}}
\newcommand{\ExtensionTok}[1]{#1}
\newcommand{\FloatTok}[1]{\textcolor[rgb]{0.00,0.00,0.81}{#1}}
\newcommand{\FunctionTok}[1]{\textcolor[rgb]{0.00,0.00,0.00}{#1}}
\newcommand{\ImportTok}[1]{#1}
\newcommand{\InformationTok}[1]{\textcolor[rgb]{0.56,0.35,0.01}{\textbf{\textit{#1}}}}
\newcommand{\KeywordTok}[1]{\textcolor[rgb]{0.13,0.29,0.53}{\textbf{#1}}}
\newcommand{\NormalTok}[1]{#1}
\newcommand{\OperatorTok}[1]{\textcolor[rgb]{0.81,0.36,0.00}{\textbf{#1}}}
\newcommand{\OtherTok}[1]{\textcolor[rgb]{0.56,0.35,0.01}{#1}}
\newcommand{\PreprocessorTok}[1]{\textcolor[rgb]{0.56,0.35,0.01}{\textit{#1}}}
\newcommand{\RegionMarkerTok}[1]{#1}
\newcommand{\SpecialCharTok}[1]{\textcolor[rgb]{0.00,0.00,0.00}{#1}}
\newcommand{\SpecialStringTok}[1]{\textcolor[rgb]{0.31,0.60,0.02}{#1}}
\newcommand{\StringTok}[1]{\textcolor[rgb]{0.31,0.60,0.02}{#1}}
\newcommand{\VariableTok}[1]{\textcolor[rgb]{0.00,0.00,0.00}{#1}}
\newcommand{\VerbatimStringTok}[1]{\textcolor[rgb]{0.31,0.60,0.02}{#1}}
\newcommand{\WarningTok}[1]{\textcolor[rgb]{0.56,0.35,0.01}{\textbf{\textit{#1}}}}
\usepackage{graphicx,grffile}
\makeatletter
\def\maxwidth{\ifdim\Gin@nat@width>\linewidth\linewidth\else\Gin@nat@width\fi}
\def\maxheight{\ifdim\Gin@nat@height>\textheight\textheight\else\Gin@nat@height\fi}
\makeatother
% Scale images if necessary, so that they will not overflow the page
% margins by default, and it is still possible to overwrite the defaults
% using explicit options in \includegraphics[width, height, ...]{}
\setkeys{Gin}{width=\maxwidth,height=\maxheight,keepaspectratio}
% Set default figure placement to htbp
\makeatletter
\def\fps@figure{htbp}
\makeatother
\setlength{\emergencystretch}{3em} % prevent overfull lines
\providecommand{\tightlist}{%
  \setlength{\itemsep}{0pt}\setlength{\parskip}{0pt}}
\setcounter{secnumdepth}{-\maxdimen} % remove section numbering

\title{Project - 1}
\author{Randall T}
\date{2/16/2021}

\begin{document}
\maketitle

\hypertarget{introduction}{%
\subsection{Introduction}\label{introduction}}

It is now possible to collect a large amount of data about personal
movement using activity monitoring devices such as a Fitbit, Nike
Fuelband, or Jawbone Up. These type of devices are part of the
``quantified self'' movement -- a group of enthusiasts who take
measurements about themselves regularly to improve their health, to find
patterns in their behavior, or because they are tech geeks. But these
data remain under-utilized both because the raw data are hard to obtain
and there is a lack of statistical methods and software for processing
and interpreting the data.

This assignment makes use of data from a personal activity monitoring
device. This device collects data at 5 minute intervals through out the
day. The data consists of two months of data from an anonymous
individual collected during the months of October and November, 2012 and
include the number of steps taken in 5 minute intervals each day.

The data for this assignment can be downloaded from the course web site:

\begin{verbatim}
Dataset: [Activity monitoring data](https://d396qusza40orc.cloudfront.net/repdata%2Fdata%2Factivity.zip)
\end{verbatim}

The variables included in this dataset are:

\begin{verbatim}
steps: Number of steps taking in a 5-minute interval (missing values are coded as NA\color{red}{\verb|NA|}NA)
date: The date on which the measurement was taken in YYYY-MM-DD format
interval: Identifier for the 5-minute interval in which measurement was taken
\end{verbatim}

The dataset is stored in a comma-separated-value (CSV) file and there
are a total of 17,568 observations in this dataset.

\hypertarget{load-dataset---data-was-downloaded-prior-to-loading}{%
\subsection{Load dataset - data was downloaded prior to
loading}\label{load-dataset---data-was-downloaded-prior-to-loading}}

\begin{Shaded}
\begin{Highlighting}[]
\KeywordTok{library}\NormalTok{(}\StringTok{"data.table"}\NormalTok{)}
\KeywordTok{library}\NormalTok{(ggplot2)}
\KeywordTok{library}\NormalTok{(tidyverse)}
\end{Highlighting}
\end{Shaded}

\begin{verbatim}
## -- Attaching packages --------------------------------------- tidyverse 1.3.0 --
\end{verbatim}

\begin{verbatim}
## v tibble  3.0.5     v dplyr   1.0.3
## v tidyr   1.1.2     v stringr 1.4.0
## v readr   1.4.0     v forcats 0.5.1
## v purrr   0.3.4
\end{verbatim}

\begin{verbatim}
## -- Conflicts ------------------------------------------ tidyverse_conflicts() --
## x dplyr::between()   masks data.table::between()
## x dplyr::filter()    masks stats::filter()
## x dplyr::first()     masks data.table::first()
## x dplyr::lag()       masks stats::lag()
## x dplyr::last()      masks data.table::last()
## x purrr::transpose() masks data.table::transpose()
\end{verbatim}

\begin{Shaded}
\begin{Highlighting}[]
\NormalTok{activitydata <-}\StringTok{ }\NormalTok{data.table}\OperatorTok{::}\KeywordTok{fread}\NormalTok{(}\DataTypeTok{input =} \StringTok{"activity.csv"}\NormalTok{)}
\end{Highlighting}
\end{Shaded}

\hypertarget{what-is-mean-total-number-of-steps-taken-per-day}{%
\subsection{What is mean total number of steps taken per
day?}\label{what-is-mean-total-number-of-steps-taken-per-day}}

\begin{enumerate}
\def\labelenumi{\arabic{enumi}.}
\tightlist
\item
  Calculate total steps per day
\item
  If you do not understand the difference between a histogram and a
  barplot, research the difference between them. Make a histogram of the
  total number of steps taken each day
\item
  Calculate and report the mean and median of the total number of steps
  taken per day
\end{enumerate}

\begin{Shaded}
\begin{Highlighting}[]
\CommentTok{#get total}
\NormalTok{total <-}\StringTok{ }\NormalTok{activitydata[, }\KeywordTok{c}\NormalTok{(}\KeywordTok{lapply}\NormalTok{(.SD, sum, }\DataTypeTok{na.rm =} \OtherTok{FALSE}\NormalTok{)), .SDcols =}\StringTok{ }\KeywordTok{c}\NormalTok{(}\StringTok{"steps"}\NormalTok{), by =}\StringTok{ }\NormalTok{.(date)]}

\CommentTok{#plot}
\KeywordTok{ggplot}\NormalTok{(total, }\KeywordTok{aes}\NormalTok{(}\DataTypeTok{x =}\NormalTok{ steps)) }\OperatorTok{+}
\StringTok{        }\KeywordTok{geom_histogram}\NormalTok{() }\OperatorTok{+}
\StringTok{        }\KeywordTok{labs}\NormalTok{(}\DataTypeTok{title =} \StringTok{"Daily Steps"}\NormalTok{, }\DataTypeTok{x =} \StringTok{"Steps"}\NormalTok{, }\DataTypeTok{y =} \StringTok{"Frequency"}\NormalTok{)}
\end{Highlighting}
\end{Shaded}

\begin{verbatim}
## `stat_bin()` using `bins = 30`. Pick better value with `binwidth`.
\end{verbatim}

\begin{verbatim}
## Warning: Removed 8 rows containing non-finite values (stat_bin).
\end{verbatim}

\includegraphics{PA1_template_files/figure-latex/unnamed-chunk-3-1.pdf}

\begin{Shaded}
\begin{Highlighting}[]
\CommentTok{#mean and median}
\NormalTok{total[, .(}\DataTypeTok{mean_steps =} \KeywordTok{mean}\NormalTok{(steps, }\DataTypeTok{na.rm =} \OtherTok{TRUE}\NormalTok{), }\DataTypeTok{median_steps =} \KeywordTok{median}\NormalTok{(steps, }\DataTypeTok{na.rm =} \OtherTok{TRUE}\NormalTok{))]}
\end{Highlighting}
\end{Shaded}

\begin{verbatim}
##    mean_steps median_steps
## 1:   10766.19        10765
\end{verbatim}

\hypertarget{what-is-the-average-daily-activity-pattern}{%
\subsection{What is the average daily activity
pattern?}\label{what-is-the-average-daily-activity-pattern}}

\begin{enumerate}
\def\labelenumi{\arabic{enumi}.}
\tightlist
\item
  Make a time series plot (i.e.~type = ``l'') of the 5-minute interval
  (x-axis) and the average number of steps taken, averaged across all
  days (y-axis)
\item
  Which 5-minute interval, on average across all the days in the
  dataset, contains the maximum number of steps?
\end{enumerate}

\begin{Shaded}
\begin{Highlighting}[]
\CommentTok{#create interval}
\NormalTok{interval <-}\StringTok{ }\NormalTok{activitydata[, }\KeywordTok{c}\NormalTok{(}\KeywordTok{lapply}\NormalTok{(.SD, mean, }\DataTypeTok{na.rm =} \OtherTok{TRUE}\NormalTok{)), .SDcols =}\StringTok{ }\KeywordTok{c}\NormalTok{(}\StringTok{"steps"}\NormalTok{), by =}\StringTok{ }\NormalTok{.(interval)]}

\CommentTok{#plot}
\KeywordTok{ggplot}\NormalTok{(interval, }\KeywordTok{aes}\NormalTok{(}\DataTypeTok{x =}\NormalTok{ interval, }\DataTypeTok{y =}\NormalTok{ steps)) }\OperatorTok{+}
\StringTok{        }\KeywordTok{geom_line}\NormalTok{() }\OperatorTok{+}
\StringTok{        }\KeywordTok{labs}\NormalTok{(}\DataTypeTok{title =} \StringTok{"Average Steps Daily"}\NormalTok{, }\DataTypeTok{x =} \StringTok{"Interval"}\NormalTok{, }\DataTypeTok{y =} \StringTok{"Average Steps per day"}\NormalTok{)}
\end{Highlighting}
\end{Shaded}

\includegraphics{PA1_template_files/figure-latex/unnamed-chunk-4-1.pdf}

\begin{Shaded}
\begin{Highlighting}[]
\CommentTok{#calc man interval}
\NormalTok{interval[steps }\OperatorTok{==}\StringTok{ }\KeywordTok{max}\NormalTok{(steps), .(}\DataTypeTok{maxint =}\NormalTok{ interval)]}
\end{Highlighting}
\end{Shaded}

\begin{verbatim}
##    maxint
## 1:    835
\end{verbatim}

\hypertarget{imputing-missing-values}{%
\subsection{Imputing missing values}\label{imputing-missing-values}}

\begin{enumerate}
\def\labelenumi{\arabic{enumi}.}
\tightlist
\item
  Calculate and report the total number of missing values in the dataset
  (i.e.~the total number of rows with 𝙽𝙰s) 2.Devise a strategy for
  filling in all of the missing values in the dataset. The strategy does
  not need to be sophisticated. For example, you could use the
  mean/median for that day, or the mean for that 5-minute interval, etc.
  3.Create a new dataset that is equal to the original dataset but with
  the missing data filled in. 4.Make a histogram of the total number of
  steps taken each day and Calculate and report the mean and median
  total number of steps taken per day. Do these values differ from the
  estimates from the first part of the assignment? What is the impact of
  imputing missing data on the estimates of the total daily number of
  steps?
\end{enumerate}

\begin{Shaded}
\begin{Highlighting}[]
\KeywordTok{nrow}\NormalTok{(activitydata[}\KeywordTok{is.na}\NormalTok{(steps),])}
\end{Highlighting}
\end{Shaded}

\begin{verbatim}
## [1] 2304
\end{verbatim}

\begin{Shaded}
\begin{Highlighting}[]
\CommentTok{# fill in missing values}
\NormalTok{activitydata[}\KeywordTok{is.na}\NormalTok{(steps), }\StringTok{"steps"}\NormalTok{] <-}\StringTok{ }\NormalTok{activitydata[, }\KeywordTok{c}\NormalTok{(}\KeywordTok{lapply}\NormalTok{(.SD, median, }\DataTypeTok{na.rm =} \OtherTok{TRUE}\NormalTok{)), .SDcols =}\StringTok{ }\KeywordTok{c}\NormalTok{(}\StringTok{"steps"}\NormalTok{)]}

\CommentTok{#create new dataset}
\NormalTok{data.table}\OperatorTok{::}\KeywordTok{fwrite}\NormalTok{(}\DataTypeTok{x =}\NormalTok{ activitydata, }\DataTypeTok{file =} \StringTok{"newdata.csv"}\NormalTok{, }\DataTypeTok{quote =} \OtherTok{FALSE}\NormalTok{)}

\CommentTok{#histogram of total with mean and median}
\NormalTok{total <-}\StringTok{ }\NormalTok{activitydata[, }\KeywordTok{c}\NormalTok{(}\KeywordTok{lapply}\NormalTok{(.SD, sum)), .SDcols =}\StringTok{ }\KeywordTok{c}\NormalTok{(}\StringTok{"steps"}\NormalTok{), by =}\StringTok{ }\NormalTok{.(date)]}

\NormalTok{total[, .(}\DataTypeTok{mean_steps =} \KeywordTok{mean}\NormalTok{(steps), }\DataTypeTok{median_steps =} \KeywordTok{median}\NormalTok{(steps))]}
\end{Highlighting}
\end{Shaded}

\begin{verbatim}
##    mean_steps median_steps
## 1:    9354.23        10395
\end{verbatim}

\begin{Shaded}
\begin{Highlighting}[]
\KeywordTok{ggplot}\NormalTok{(total, }\KeywordTok{aes}\NormalTok{(}\DataTypeTok{x =}\NormalTok{ steps)) }\OperatorTok{+}
\StringTok{        }\KeywordTok{geom_histogram}\NormalTok{() }\OperatorTok{+}
\StringTok{        }\KeywordTok{labs}\NormalTok{(}\DataTypeTok{title =} \StringTok{"Dailt Steps"}\NormalTok{, }\DataTypeTok{x =} \StringTok{"Steps"}\NormalTok{, }\DataTypeTok{y =} \StringTok{"Frequency"}\NormalTok{)}
\end{Highlighting}
\end{Shaded}

\begin{verbatim}
## `stat_bin()` using `bins = 30`. Pick better value with `binwidth`.
\end{verbatim}

\includegraphics{PA1_template_files/figure-latex/unnamed-chunk-5-1.pdf}

\hypertarget{are-there-differences-in-activity-patterns-between-weekdays-and-weekends}{%
\subsection{Are there differences in activity patterns between weekdays
and
weekends?}\label{are-there-differences-in-activity-patterns-between-weekdays-and-weekends}}

\begin{enumerate}
\def\labelenumi{\arabic{enumi}.}
\tightlist
\item
  Create a new factor variable in the dataset with two levels --
  ``weekday'' and ``weekend'' indicating whether a given date is a
  weekday or weekend day.
\item
  Make a panel plot containing a time series plot (i.e.~𝚝𝚢𝚙𝚎 = ``𝚕'') of
  the 5-minute interval (x-axis) and the average number of steps taken,
  averaged across all weekday days or weekend days (y-axis). See the
  README file in the GitHub repository to see an example of what this
  plot should look like using simulated data.
\end{enumerate}

\begin{Shaded}
\begin{Highlighting}[]
\CommentTok{#making new variable}
\NormalTok{new_activity <-}\StringTok{ }\NormalTok{activitydata }\OperatorTok
\StringTok{        }\KeywordTok{mutate}\NormalTok{(}\DataTypeTok{day_of_week =} \KeywordTok{weekdays}\NormalTok{(}\DataTypeTok{x =}\NormalTok{ date))}

\CommentTok{#creating variable into factor}
\NormalTok{new_activity[}\KeywordTok{grepl}\NormalTok{(}\DataTypeTok{pattern =} \StringTok{"Monday|Tuesday|Wednesday|Thursday|Friday"}\NormalTok{, }\DataTypeTok{x =} \StringTok{`}\DataTypeTok{day_of_week}\StringTok{`}\NormalTok{), }\StringTok{"weekday/weekend"}\NormalTok{] <-}\StringTok{ "weekday"}

\NormalTok{new_activity[}\KeywordTok{grepl}\NormalTok{(}\DataTypeTok{pattern =} \StringTok{"Saturday|Sunday"}\NormalTok{, }\DataTypeTok{x =} \StringTok{`}\DataTypeTok{day_of_week}\StringTok{`}\NormalTok{), }\StringTok{"weekday/weekend"}\NormalTok{] <-}\StringTok{ "weekend"}

\NormalTok{new_activity[, }\StringTok{`}\DataTypeTok{weekday/weekend}\StringTok{`} \OperatorTok{:}\ErrorTok{=}\StringTok{ }\KeywordTok{as.factor}\NormalTok{(}\StringTok{`}\DataTypeTok{weekday/weekend}\StringTok{`}\NormalTok{)]       }

\CommentTok{#create panel plot                }
\NormalTok{new_activity[}\KeywordTok{is.na}\NormalTok{(steps), }\StringTok{"steps"}\NormalTok{] <-}\StringTok{ }\NormalTok{new_activity[, }\KeywordTok{c}\NormalTok{(}\KeywordTok{lapply}\NormalTok{(.SD, median, }\DataTypeTok{na.rm =} \OtherTok{TRUE}\NormalTok{)), .SDcols =}\StringTok{ }\KeywordTok{c}\NormalTok{(}\StringTok{"steps"}\NormalTok{)]        }
\NormalTok{interval <-}\StringTok{ }\NormalTok{new_activity[, }\KeywordTok{c}\NormalTok{(}\KeywordTok{lapply}\NormalTok{(.SD, mean, }\DataTypeTok{na.rm =} \OtherTok{TRUE}\NormalTok{)), .SDcols =}\StringTok{ }\KeywordTok{c}\NormalTok{(}\StringTok{"steps"}\NormalTok{), by =}\StringTok{ }\NormalTok{.(interval, }\StringTok{`}\DataTypeTok{weekday/weekend}\StringTok{`}\NormalTok{)] }

\KeywordTok{ggplot}\NormalTok{(interval, }\KeywordTok{aes}\NormalTok{(}\DataTypeTok{x =}\NormalTok{ interval, }\DataTypeTok{y =}\NormalTok{ steps, }\DataTypeTok{color =} \StringTok{`}\DataTypeTok{weekday/weekend}\StringTok{`}\NormalTok{)) }\OperatorTok{+}
\StringTok{        }\KeywordTok{geom_line}\NormalTok{() }\OperatorTok{+}
\StringTok{        }\KeywordTok{labs}\NormalTok{(}\DataTypeTok{title =} \StringTok{"Average Steps by type of day"}\NormalTok{, }\DataTypeTok{x =} \StringTok{"Interval"}\NormalTok{, }\DataTypeTok{y =} \StringTok{"# of Steps"}\NormalTok{) }\OperatorTok{+}
\StringTok{        }\KeywordTok{facet_wrap}\NormalTok{(}\OperatorTok{~}\StringTok{`}\DataTypeTok{weekday/weekend}\StringTok{`}\NormalTok{, }\DataTypeTok{ncol =} \DecValTok{1}\NormalTok{, }\DataTypeTok{nrow =} \DecValTok{2}\NormalTok{)}
\end{Highlighting}
\end{Shaded}

\includegraphics{PA1_template_files/figure-latex/unnamed-chunk-6-1.pdf}

\end{document}
